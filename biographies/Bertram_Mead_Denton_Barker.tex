\biohead{Bertram Mead Denton Barker}{During the war\cite{BMDBwar}}

Bertram Mead Denton Barker was born on 13 February 1915 in Birkenhead, Cheshire.  His parents were James Denton Barker (\p{James_Denton_Barker}) and Kathleen Munday (\p{Kathleen_Munday}) and he had two siblings: Ralph Munday Denton Barker (\p{Ralph_Munday_Denton-Barker}) and Virginia Kathleen Denton Barker (\p{Virginia_Kathleen_Denton_Barker}).

Known by his second name, Mead, he was educated at Cheam and Felsted Schools, and then trained as a Mechanical Engineer. He served as a pilot in the RAF during the war,  after training in Texas (1942--1943) at the Terrell Aviation School and then at the British Flying Training School in 1943 where he  received recognition as the best cadet:  as shown by the inscription on his cigarette case which read as follows: ``Presented by Major W.F.Long, Terrell Aviation School to L.A.C. B.M.D. Barker as the best all round cadet of the Tenth Course at No. 1 British Flying Training School 1st January 1943.''

After demobilisation he was employed as an engineer in the Midlands. He married Charlotte Marion Rabus (\p{Charlotte_Marion_Rabus}) on 18th March 1948 at the Marylebone Presbyterian Church \cite{TheTimes1948-03-22} and had one daughter (Rosalie).

He died on 30 August 1980, in Solihull, Warwickshire. 

An obituary written by close colleague Roy Beebee reads:

\begin{quotation}
Anyone listening out on the right frequency near Dallas, Texas one day in the early nineteen forties might have heard an RT conversation which went something like this:

``Tower, this is X-ray Fox Seven Niner solo, down wind, wheels down, locked landing. Over.''

``Seven-niner from Tower did you say solo? Over''

``Tower from seven-niner affirmative my instructor has made alternative arrangements---by parachute. Out.''

The cadet Pilot was Mead Barker.

Only Mead could have convinced the Establishment that his instructor's action was not through panic and go on to win the award for the Most Outstanding Cadet of his course.

Mead Barker died on Friday, 29th August 1980 after a year long distressing illness. He was 65 but most people will remember him as a seemingly much younger enthusiastic Talbot owner with a depth of absorbing knowledge on a wide variety subjects which could be readily plumbed by anyone who had the good fortune to converse with him.

Whatever he had to say was of interest and usually it was not long before his amusing turn of phrase resulted in dialogue of dry mirth.

Always a perfectionist his magnum opus was the concours winning rebuild of the 1930 500 mile race single seater Works Talbot 90 GX68, back to the two seater road car form it was in 1934 when it was owned by Hebler.

Typical of Mead's attention to detail were the visits he made to Roesch, to Hebler and to other previous owners of the car in order to verify certain features.

Typical too of Mead was his willingness to spend considerable time helping others even when in the midst of this exercise of dedication.

Not so well known were his other wide interests which included model making, classical music, fell walking and clock making; to all of these he applied himself with considerable skill. He possessed a prodigious memory and could shame continentals with the accuracy of his interesting knowledge of their history.

His entire working life was involved with engineering until he took an early retirement (to finish the Talbot?). Latterly he had been Works Director at Enots Ltd. where he had worked for most of the post war period, apart from a short spell with the Dunlop Rubber Company which, after the war, brought him back to earth.

Prior to the period in the RAF he had been apprenticed at Camel-Laird and worked at the Bristol Aircraft Company and Leyland Motors. He was educated at Cheam and Felstead and was a native of Birkenhead where his father was an Average Adjuster.

Mead's amusing and always interesting conversation plus his infectious laugh will be much missed by all who knew him. He leaves a widow and daughter and family to whom we extend our sympathy in their loss."
\end{quotation}
