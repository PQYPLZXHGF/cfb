\biohead{John Hill}{}

John Hill was born in 1729 in Paulton, Somerset, to Joseph Hill (1700--1749) and Mary, whose family name is unknown (?--1759). He had three siblings: Joseph Hill (1727--1767), Robert Hill (1731--1787), and Elizabeth Hill.

He married Elizabeth Annie Ames in 1751 and they had ten children: Simon Hill (1752--1814), Joseph Hill (1755--1782),
Thomas Ames Hill (1758--1827), Elizabeth Hill (1760--1781), Hepzibah Hill, Elizabeth Hill (1762--?), Susanna Hill (1765--?), John Hill (1767--1796), George Hill (\p{George_Hill}), and Robert Hill (1775--1839).

He was a Coalmaster, and also owned an Inn in Paulton, Somerset and lived at Hill House, Paulton. A description of  Hill House, as written in a Document of land transfer (held by a living family member) reads:

The earliest available deeds of the property refer to the land being bought by John Hill Gentleman of the Parish of Paulton from Robert Jeanes, Yeoman of the Parish of Priston, ``two closes of meadow or pasture Ground adjoining together called the Meads. Containing by estimation two acres be it more or less situate adjoining to a dissenting house commonly called the Baptists. One other close of meadow or pasture Cround above the said Meeting house containing by estimation two acres and a half be it more or less which said closes of Ground are now in the possession of John Gregory as tenant to the said Robert Jeanes.'' Deeds dated the eighteenth day of October in the thirty-fourth year of the reign of our Sovereign Lord George the Second, which is 18 October 1760.

In 1767 John Hill bought more land from the descendants of Joseph Padfield and paid twenty-four pounds and ten shillings. In the deeds he is described as an Innkeeper. He improved the dwelling that stood on the site, by building the Georgian house in the front of an existing cottage. The existence of a large malt house on the land made it an attractive proposition. Legend has it that John Wesley preached from the steps of this building in September 1765. Within this building there is evidence that it was used for malting barley; there is a soaking pit next to a well and drain by the old entrance.

John Hill also bought a considerable amount of land in Paulton including Pearce's Stile, adjoining the orchard of Hill House.

Further information about his coal-mining properties are noted in the following documents (held by living family member):

26 February 1779: Transfer of land in Paulton, for coal mining---Deeds signed by John Hill, Elizabeth Palmer and Robert Hill.

25 March 1768: Indenture of Assignment between John Hill of Paulton Innholder and Robert Hill of same, Butcher:
Within this property are: Tools, tackling and things for mining gaining raising and landing the said coals - together with carts, carriages and otherwise to remove take and carry away the same.
Under the lands comprised in the above demesne were valuable veins or beds of coal and they have been worked from a period previous to the year 1768 by a Company of Proprietors called 'The Paulton Coal Company' of which all the above named and since their deaths their legal representatives are partners.
The customary mode of mining for coal in Somerset is for parties in the Works to pay to the proprietor or lessor of the land from which the coal is taken an eighth or tenth or some other proportion of the coal.
Joseph Hill under the will of Joseph Hill of 1767 assumed to be entitled to the coal under the lands above mentioned calling himself the Heir at law of John Hill the grantor by the deal of 1697 for the 1000 years next\dots."

He died on 10 January 1789 in Paulton, Somerset aged 60 and was buried on 15 January 1789 in Paulton Churchyard. The inscription on his tomb reads:

\begin{quotation}
In memory of John Hill of this parish who died January 10th 1789 aged 60 years.

Also Elizabeth wife of the above who died July 6th 1806 aged 75 years.

Also Betty daughter of the above who died March 25th 1781 aged 20.

Also of Joseph their son who died November 27th 1782 aged 27.

Also of John their son who died July 2nd aged 29.

Also of Simon their son who died December 3rd 1814 aged 62 years.

Also Mary wife of Thomas Ames Hill who died May 2nd 1822 aged 64 years.

Also Thomas Ames Hill son of John and Elizabeth Hill who died August 18th 1827 aged 69.

Also of Robert son of John and Elizabeth Hill who died November 25th 1839 aged 65.

Also Mary wife of Robert Hill who died January 13th 1843 aged 70.
\end{quotation}

The following information was taken from John Hill's will, in Memorandum, and was  written some years after his death with regard to the ownership of the Radstock Coal Mine, which gives information about his descendants: ``The following statement is an explanation of the names and division in which all the interests of the late John Hill (Gentleman) of Paulton had in the Radstock Coal works which since became the shares and holdings of such respective members of his family as are here stated. The late Mr John Hill of Paulton Gentleman aforesaid dying without a will his share and interest in the Radstock coal works at his decease became divided in the following manner (viz): His widow Mrs Betty Hill became entitled to one third, and the other two thirds became divided in the following manner between his eight children (namely) Simon Hill, Thomas Ames Hill, John Hill, George Hill, Robert Hill, Susannah Hill (Mrs James), Mary Hill (Mrs Broddribb), and Hepzibah Hill (Mrs Parsons) share and share alike. And Mrs Betty Hill aforesaid at her decease left by will her third part to be divided into equal parts and given to the following individuals (namely) Thomas Ames Hill, George Hill, Robert Hill, Mrs James, Betsy daughter of Mrs Brodribb now Mrs Short and the remaining sixth part to the children of her daughter Mrs Parsons (viz) Maria (now Mrs Dudden), Caleb, William and Elizabeth (late Mrs Pope). John Hill aforesaid at his decease gave his share of one eighth to his nephew John Hill James, second son of Mr James, and Mrs James at her decease gave her share of one eighth and her share of one sixth to her two sons Thomas and? to be equally divided between them and Simon Hill aforesaid at his decease gave his share of one eighth to the said Thomas and John Hill James to be equally divided between them/ And Thomas Ames Hill aforesaid at his decease gave his share of one eight and his share of the sixth to his nephew Thomas Ames Hill, son of George Hill aforesaid and the same said George Hill at his decease left his share of one eighth and his share of one sixth to his wife Mrs Hannah Hill. And the one eighth share of the late Mrs Parsons at her decease became the property of her husband Mr Jonathon Parsons. And Caleb Parsons son of Mrs Parsons gave his share to his brother [illegible].'' (The document is held by a living family member.)
