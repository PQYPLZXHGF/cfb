\biohead{Thomas Henry Barker}{}

Thomas Henry Barker was born on 18 May 1841\cite{THBbirth} at 15 Earle Street in Liverpool,\cite{THBplantagenets} the second son of Charles Frederick Barker (\p{Charles_Frederick_Barker}) and Elizabeth Hazelwood (\p{Elizabeth_Hazelwood}), and he was baptised on 8 June 1841 at St Peters, Liverpool \cite{THBbaptism}.  His siblings were Charles Frederick Barker (1838--1887),  Elizabeth Barker (1838--1840) and Joseph Bolton Barker (1844 --?). 

Thomas Henry was living at 79 Canning Street, Liverpool in 1861 at the age of 19, with his mother Elizabeth (already a widow) and worked as a ship owners clerk.  In the household were his brother Joseph Bolton Barker (16), brother Charles N. Barker (age 24) Mariner and their aunt Isabella Hazelwood, age 52.

In the 1871 census he was still at Canning Street and worked as a Merchant. After this he lived at 12 Norwood Grove\cite{THBcensus}.


On 25 August 1875 he married [[Mary Ellen Moulsdale]]\cite{THBmarriage} with whom he would have seven sons:\cite{THBplantagenets}  James Denton Barker (\p{James_Denton_Barker}), Charles Frederick Strangways Barker (\p{Charles_Frederick_Strangways_Barker})), Reverend Thomas Percy Conyers Barker (\p{Thomas_Percy_Conyers_Barker}), Francis Darcy Mead Barker (\p{Francis_Darcy_Mead_Barker}), William Danby Holt Barker (\p{William_Danby_Holt_Barker}), 
Jonathan Tong Barker (\p{Jonathan_Tong_Barker}) and Henry Bertram Mitford Barker (\p{Henry_Bertram_Mitford_Barker}).
 
They were living at 10 Falkner Street, Liverpool in July 1876 and his occupation was a coal merchant [TFBoccupation];  by 1891 they had moved to Edge Lane, Liverpool\cite{THBresidence} and then in 1903 they lived at ``Ormesby'', 42 Brookfield Gardens, West Kirby, Cheshire (where he lived until his death).\cite{THBdeathcert}

He became the Secretary of the Liverpool Chamber of Commerce on 15 August 1884 and was Secretary for 26 years.  He was very active with the Chamber and was heavily involved in promoting the industry and trade of the city.  He was presented with a bound book of speeches and writings and a large portrait on 30 April 1906.

He travelled extensively overseas, and promoted Liverpool trade with West Africa, America and Russia, and collaborated in setting up the Department of Russian Studies at the University of Liverpool and his entry in  the 1907 \emph{Who's Who} reads:

"For 26 years Secretary of the Incorporated Chamber of Commerce of Liverpool: b. Liverpool 18 May 1841, son of Charles Frederick Barker of Copenhagen, m. Mary Ellen d. of John Moulsdale of Liverpool. Educ. at private schools and Queens College Liverpool. Received early business training in two of the largest Shipping and East India houses in Liverpool. Afterwards travelled extensively in Europe, North Africa, America and Australasia. Four years ago toured Northern Hemisphere, via Canada, Japan, North China, Cerea, Manchuria, Siberia, Russia. Member of (1) Council Liverpool Geographical Society since its formation; (2)Committee of City of Liverpool School of Commerce, and (3) of Liverpool Committee of Trinity College, London; also of Surtees Society, Yorkshire Archaeological Society, Yorkshire Parish Register and Thoresby Societies. Author of many widely circulated Reports on Railway legislation, facilities and Schemes, Charges of the Port of Liverpool, Affairs of West Africa, including Niger, Affairs of the East and other subjects relating to Commerce. Member of Constitutional and Granville Clubs, London. Recreations: literature, British and Foreign, the fine arts, archaeology, \&c. Business address B10, Exchange Buildings, Liverpool. Residence: "Ormesby", West Kirby, Cheshire."

He was presented with a bound book of speeches and writings and a large portrait on 30 April 1906 in recognition of his achievements,  and the speech given at that occasion is as follows:\cite{THBbio}

\begin{quotation}
Mr Barker joined the Liverpool Chamber of Commerce as Assistant Secretary in 1881. In 1884 he succeeded Mr William Blood as Secretary.

Mr Barker formed in that year the African Trade Section whose work has led to a great extension of British Possessions and interests in West Africa. In this connection mention may be made of "the large increase in Imports of West African timber, to the arrangement for Timber Concessions and the fixing of Royalties and other charges at the lowest obtainable figure.

In 1886 Mr Barker travelled over the Canadian Pacific Railway from Quebec to Vancouver, shortly after its opening, in order to see and report upon the commercial resources of the newly opened parts of the Dominion. From 1888, when the Railway and Canal Traffic Act 1888 was passed, he worked extensively in the matters of Railway Reform, including Reclassification of goods and reduction of rates.

Mr Barker also drew up important reports upon the Effects on the Port of Liverpool of the opening of the Manchester Ship Canal, and on the Administration and Charges of the Port. These reports were presented to the Mersey Docks \& Harbour Board and their recommendations largely adopted by the Board The result was substantial reductions in Rates \& Dues. These matters being of vital interest to the Timber Trade, the Association was represented upon the Committees by the late Messrs J Berkeley Smith, and James Harrison and, later, by Alderman James Webster.

Mr Barker organised and represented the Chamber on a large number of Deputations to Government Departments on many matters affecting the trade of the country and of the Port of Liverpool. Between 1884 and the present time Mr Barker wrote more than 100 Memorials which were presented to various Departments of the State and which may be classified as follows, namely 55 Memorials on Home Administration and Legislation, 25 on Indian, Colonial and Foreign subjects and 20 on African subjects. Also 20 special reports were drawn up and presented to successive Governments, in addition to reports on Parliamentary Bills.

In the autumn of 1903 Mr Barker attended the Meeting of Chambers of Commerce of the Empire at Montreal, when he took the opportunity of again crossing Canada by the newest route; thence visiting Japan, North China, Corea, Manchuria and Siberia, travelling over the Trans Siberian Railway from Dalny to Moscow, in order to report upon prospects of extension of British Trade with Siberia \&c.

Some ten years ago a Russian Section was added to the Chamber.

Mr Barker was one of a small Deputation, including the late Sir Alfred Jones, which waited upon the Czar of Russia at Cowes, subsequent to the visit of members of the Duma to Liverpool and was made Chevalier of the Order of Saint Anne of Russia.
\end{quotation}

Thomas Henry died on 9 April 1917\cite{THBdeathcert} and the cause of death was given as:

 1. Acute Prostatatis 2--3 days. 2. Cystitis pneumonia 3 days---informant T. P. Conyers Barker, son).  He was buried at the Smithdown Cemetery, Liverpool on 12 April 1917.\cite{ToxtethBarker20}
 
 The funeral notice in the The Liverpool Courier, Thursday April 12 1917 read:
``Barker April 9 at Ormesby, West Kirby in his 76th year, Thomas Henry Barker the beloved husband of Mary Ellen Barker for many years Secretary of the Liverpool Chamber of Commerce. Interment at Smithdown Cemetery today (Thursday) at 2.30pm. (Friends kindly accept this, the only intimation).''
