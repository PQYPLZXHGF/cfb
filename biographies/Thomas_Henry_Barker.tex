\biohead{Thomas Henry Barker}{ThomasHenryBarker}{}

Thomas Henry Barker was Secretary of the Liverpool Chamber of Commerce for 26 years.\footnote{Typewritten biography of Thomas Henry Barker}

He was born on 18 May 1841\footnote{General Registry Office, birth certificate BXCF518186} at 15 Earle Street in Liverpool,\footnote{Descent of Thomas Henry Barker from the Plantagenets} the second son of Charles Frederick (\p{Charles_Frederick_Barker}) and Elizabeth Barker (\p{Elizabeth_Hazelwood}).

Thomas Henry lived at 79 Canning Street, Liverpool in 1861 at the age of 19, with his mother Elizabeth (already a widow) and worked as a ship owners clerk.  In the household were his brother Joseph Bolton Barker (16), brother Charles N. Barker (age 24) Mariner and their aunt Isabella Hazelwood, age 52.

In the 1871 census he was still at Canning Street and worked as a Merchant. On 25 August 1875 he married [[Mary Ellen Moulsdale]],\footnote{Marriage certificate of Thomas Henry Barker and Mary Ellen Moulsdale, 25 August 1875. \url{https://www.flickr.com/photos/freosam/15441758029/}} with whom he would have seven sons:\footnote{Descent of Thomas Henry Barker from the Plantagenets}
in 1876, James Denton Barker (\p{James_Denton_Barker});
1878, Charles Frederick Strangways Barker (\p{CharlesFrederickStrangewaysBarker});
1879, Thomas Percy Conyers Barker (\p{ThomasPercyConyersBarker});
c. 1880, Francis Darcy Mead Barker (\p{FrancisDarcyMeadBarker});
1882, William Danby Holt Barker (\p{WilliamDanbyHoltBarker});
1883, Jonathan Tong Barker (\p{JonathanTongBarker}); and in
1885, Henry Bertram Mitford Barker (\p{HenryBertramMitfordBarker}).

He became the Secretary of the Liverpool Chamber of Commerce on 15 August 1884.  He was very active with the Chamber and was heavily involved in promoting the industry and trade of the city.  He was presented with a bound book of speeches and writings and a large portrait on 30 April 1906.(see below)

He travelled extensively overseas, and promoted Liverpool trade and merchants with West Africa and with Russia, and collaborated in setting up the Department of Russian Studies at the University of Liverpool.

He may have died shortly before 10 April 1917.\footnote{\emph{Liverpool Echo} newspaper, 10 April 1917.}\footnote{FreeBMD record, retrieved 31 August 2014. GRO entry: Wirral 8a/504. \url{http://freebmd.org.uk/cgi/information.pl?r=137462560&d=bmd_1407157232} \emph{Deaths June 1917: Barker, Thomas H. (age 75)}}

The following is a short biography of TH Barker, probably written not long after his death.
\footnote{
	\emph{Biography of Thomas Henry Barker}, author unknown, c.\ 1920.
	\url{https://en.wikisource.org/wiki/Biography_of_Thomas_Henry_Barker}
}

\begin{quotation}
Mr Barker joined the Liverpool Chamber of Commerce as Assistant Secretary in 1881. In 1884 he succeeded Mr William Blood as Secretary.

Mr Barker formed in that year the African Trade Section whose work has led to a great extension of British Possessions and interests in West Africa. In this connection mention may be made of "the large increase in Imports of West African timber, to the arrangement for Timber Concessions and the fixing of Royalties and other charges at the lowest obtainable figure.

In 1886 Mr Barker travelled over the Canadian Pacific Railway from Quebec to Vancouver, shortly after its opening, in order to see and report upon the commercial resources of the newly opened parts of the Dominion. From 1888, when the Railway and Canal Traffic Act 1888 was passed, he worked extensively in the matters of Railway Reform, including Reclassification of goods and reduction of rates.

Mr Barker also drew up important reports upon the Effects on the Port of Liverpool of the opening of the Manchester Ship Canal, and on the Administration and Charges of the Port. These reports were presented to the Mersey Docks \& Harbour Board and their recommendations largely adopted by the Board The result was substantial reductions in Rates \& Dues. These matters being of vital interest to the Timber Trade, the Association was represented upon the Committees by the late Messrs J Berkeley Smith, and James Harrison and, later, by Alderman James Webster.

Mr Barker organised and represented the Chamber on a large number of Deputations to Government Departments on many matters affecting the trade of the country and of the Port of Liverpool. Between 1884 and the present time Mr Barker wrote more than 100 Memorials which were presented to various Departments of the State and which may be classified as follows, namely 55 Memorials on Home Administration and Legislation, 25 on Indian, Colonial and Foreign subjects and 20 on African subjects. Also 20 special reports were drawn up and presented to successive Governments, in addition to reports on Parliamentary Bills.

In the autumn of 1903 Mr Barker attended the Meeting of Chambers of Commerce of the Empire at Montreal, when he took the opportunity of again crossing Canada by the newest route; thence visiting Japan, North China, Corea, Manchuria and Siberia, travelling over the Trans Siberian Railway from Dalny to Moscow, in order to report upon prospects of extension of British Trade with Siberia \&c.

Some ten years ago a Russian Section was added to the Chamber.

Mr Barker was one of a small Deputation, including the late Sir Alfred Jones, which waited upon the Czar of Russia at Cowes, subsequent to the visit of members of the Duma to Liverpool and was made Chevalier of the Order of Saint Anne of Russia.
\end{quotation}


