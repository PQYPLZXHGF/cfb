\biohead{Jemima Browne}{}

Jemima Browne was born in	1770 	to Benjamin Browne and Sarah Harris. She had three siblings: Benjamin Browne, John Browne and Katherine Browne.

She married James Munday (\p{James_Munday}) on 15 February 1798 at Bishopstrow Church, Wiltshire and they had nine children (see\p{James_Munday})

She died on 	27 May 1839 and in her will (the original of which is held by a living family relative) she writes as follows:

"I desire to be buried in the most plain manner and to have an oak coffin with no ornaments with only name, age and date of the year, to have no shroud but a clean nightgown which one of my dear daughters will be so kind as to see put on not for me to be exposed to strangers. I should like to been buried in the same grave with my dear beloved husband but as I am so far removed from the spot that contains his dear remains I desire to be interred in the place wherein I may not have walls on my grave but a flat stone laid over me with my name and age. It is my earnest request that all just ?manners? may be discharged that no-one may be injured by me. I give and bequeath to my dear son William Munday the Gold Watch and his late lamented Father's Bible which I desire may never go out of our family, but be his son's property and never be sold. I give to my dear son Henry Thomas Munday one feather bed, bolster and pillows and mahogany chest of drawers and one mahogany table standing, and damask table cloth, a pair of silver tablespoons. And to my dear daughter Catherine Munday I give my workbox which was promised to me by Mrs Temple, and a gold ring to the memory of my dear beloved father. Also to my dear daughter Sarah Munday I give a pair of silver tablespoons and a gold ring set with pearls to the memory of my own ever beloved mother. Also to my dear daughter Jemima Harris I give a gold ring that was left to the memory of the late Mrs. Butt (?). Also to my dear son James Munday the two engravings of the spirit of a child carried to Heaven by an angel. The reason of me not leaving him spoons was that I gave them to him when he was married and all the rest of my plates I bequeath to my dear daughter Mary Elizabeth Munday for her sole use. For other property, but providing there should be any left after my funeral expenses and debts are paid, I would wish to be parted equally between all my children."
