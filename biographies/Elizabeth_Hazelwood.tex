\biohead{Elizabeth Barker (n\'{e}e Hazelwood)}{}

Elizabeth was born on 5 April 1807 \cite{ElizabethHazelwoodBirth} in Whitby, Yorkshire, to  Moses Hezelwood (\p{Moses_Hazelwood}) and Elizabeth Meade (\p{Elizabeth_Meade}).  She had seven siblings: Mary Hazelwood (1805--1887), Isabella Hazelwood (1808--1882), Sarah Hazelwood (1811--?), Francis Mead Hazelwood (1813--?), Thomas Hezelwood (1814--1851), Francis Hazelwood (1816--?) and Trufit Mead Hazelwood (1817--?). 

She married Charles Frederick Barker on 3 February 1836 at St Dunstan's, in Stepney, Middlesex \cite{ElizabethHazelwoodMarriage} (although according to her brother Thomas in his notebook, held by a family member, they had left Whitby together to live in Stepney in 1834: ``My sister and Barker left Whitby on September 23\textsuperscript{rd} 1834.'')  They had three sons (one daughter died in infancy):  Charles Frederick Barker (1836--1887), Thomas Henry Barker (\p{Thomas_Henry_Barker}) and Joseph Bolton Barker (1844--?).

On 18 May 1841 they were living at 9 Earle Street, Liverpool \cite{ElizabethHazelwoodResidence}, and by 1851 they were in Toxteth Park, Liverpool, Lancashire. In July 1875 she had moved to	Peckham, Surrey,15 Ryder Villas, St Mary's Road, Peckham, Surrey to live with her youngest son, Joseph: letters from Elizabeth (held by family member) in 1875 to her daughter-in-law (married to Thomas Henry, known as Tom) show that she was living with her son Joseph (Joe) and she welcomes Mary into the family. She also enquires into the health of Mrs. Denton (Mary's aunt). In 1878 she writes about her sister Maria, who is living with Tom and Mary. In 1881, Elizabeth had moved to live (with her son Joe) in Streatham. 

Shortly before her death she wrote the following letter with regard to her private property: (On an envelope addressed by Mrs Barker, 2 Ryden Villas, Rossiter Rd, Balham):

\begin{quotation}
My dear children Charles Tom and Joe I have for a long time thought of putting down on paper my wishes with regard to the few things I posess (sic) . There is not much of value only for the sake of them having belonged to your dear Father and Mother. I cannot make an equal distribution as Joe’s house has so long been my home that I consider he ought to have xxx in the first place. I should (line through) wish him to have the things in my bedroom, that is bedstead bed bedding drawers washstand dressing table chairs \& carpet and glass --- there are a few things of your dear Fathers bringing I should like you each to have one of the two large vases china dish and stand and the bamboo ornaments and small vases --- beside many little things. I cannot name my books I wish Charles to have Fletchers family devotion Tom Pilgrims progress Joe Sundays at home and divide according to your own judgement Tom gave me many of them and can choose for himself the one over the dining room mantle piece is the only one of value. Tom can have his oil paintings if he xxx Mr Birkett’s oil paintings xxxxxxx Tom always thought he had a right to them these things I must leave to your own judgment as(?) with regard to bed linen what I have is nearly worn if you would like to divide it My clothes whatever would be useful to my sister if she survives me I wish her to have The rest divide as you like and let it all be done peaceably my ?? only the brooches the larger with your dear Father’s hair. I wish Charles to have for Barbara the amythest. And the little pe... that was Mrs ,,,,,,,,,,, Tom to have for Mary, and a small black one Joe for Millie
My old watch for Ida and the little seal and key for Hilda my chain I should like cut in two and half for Harry and half for Jimmie when old enough they could dispose of it to go toward buying........"
\end{quotation}

Held in personal papers.

Elizabeth died on 17 December 1882 Liverpool, Lancashire at 134 Windsor Street, Liverpool and was buried on 24 December at the Anfield Cemetery, 
