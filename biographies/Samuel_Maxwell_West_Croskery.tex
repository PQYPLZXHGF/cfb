\biohead{Samuel Maxwell West Croskery}{Taken in Yokohama, Japan in 1905 and sent to his daughters.\cite{SMWCYokohama1905}}

Samuel Maxwell West Croskery was born in 1847 in Downpatrick, Co. Down, Ireland, to Hugh Croskery (\p{Hugh_Croskery}) and Charlotte Wallace Brown (\p{Charlotte_Wallace_Brown}).\cite{SMWC-MG-marriage}. He had seven siblings: Hugh Croskery (1835--1886), Ann Croskery (1836--1931), Alexander Brown Croskery (1838--1897), Albert James Croskery (1840--1865), Horatio Collingwood Croskery (1842--1929), Frederick C. Croskery (1845--?), and Wallace Brown Croskery (1851--1926).

He married Mary Gilmour (\p{Mary_Gilmour}) on 13 August 1874 in Troon (Ayrshire), when he was living in D\`{u}n Laoghaire, County Dublin, Ireland.\cite{SMWC-MG-marriage} They had two daughters: Jeanie Elenora Dunsmuir Croskery(known always as Nora) (\p{Jeanie_Elenora_Dunsmuir_Croskery}) and Marian Gilmour Croskery (\p{Marian_Gilmour_Croskery}).

Samuel became a Second Mate in Liverpool on 20 September 1869. There is an extensive record of all his subsequent voyages as Master, in Lloyds Registers:\cite{MarineRecordsSMWC} From 1869 onwards, he sailed to Australia, New York, Delaware, Nova Scotia, Singapore, Napier and Wellington (NZ), San Francisco, and Calcutta. On one of his voyages he rescued the crew of a stricken ship, the Benlarig, and the following was published on the front page of \emph{The Morning Call} in San Francisco on 23 February 1895:

\begin{quotation}
\textsc{Blown to Sea in a Blizzard.}

\emph{Terrible Experience of the British Ship Benlarig.}

Baltimore, Feb. 22.---The steamer Rossmore arrived to-day with Pilot Franklin Beebe of New York and news of the overdue ship Benlarig, which left Caleta Buena, Chile, October 6, with a cargo of niter for New York.

She was seventy-five miles off New York February 5, when she took Pilot Beebe aboard to guide her into New York. Two days afterward the blizzard carried her to sea. All her sails were blown away. One of the crew was thrown and had a leg broken, and the intense cold prostrated three more with frost-bitten limbs. Two seamen died. The ship's company were put on short rations. After fourteen days' tossing about in the blizzard, the Rossmore, from Liverpool to Baltimore, sighted the ship on Monday night 130 miles off Sandy Hook. The Rossmore stopped and a boat put off from the distressed ship. Pilot Beebe was almost prostrated with illness. Captain Beall and seamen of the Benlarig refused to leave the ship. Captain Croskery supplied the ship's boat with food sufficient to last ten days. 
\end{quotation}

He refers to this episode in the  following letter, which he wrote to his sister-in-law Mary Anne Mortimer Thomson in New Zealand shortly after the death of his brother Alexander Brown Croskery:\cite{SMWCletter}
\begin{quotation}
\begin{flushright}
S. S. ``Rosmore'' \\
at sea \\
5\textsuperscript{th} June 1897
\end{flushright}

Dear Minnie,

I must tell you how shocked I was to hear of poor Alex's death. When I got home last voyage I had written you such a gossipy letter before, when at sea, I had let it go on, for I have such a short time in port that really I have not one minute to spare when in Liverpool. Last time only 54 hours so you see how quickly we are moved around.

Mary was very sorry. She always liked Alexander more than any of my brothers. He had such a kindly nature with him. Nora sent on your letter to Wallace at Eckington, and he sent word to Father. Poor old man he will I fear soon follow his Son. I have not seen him now for three years but hope to this fall. I fully expect that Mary and the two girls, Nora and Marion, will cross over to Dublin, when I return next to stay there for a month. I am sorry to say Mary is very far from well. Her heart has been giving her a lot of trouble as also a rupture of the navel, and being very stout, as you know, its very bad for her. However I hope the change, and at the sea shore, Bray or Dalthy, will do her good for she is a dear good wife to me, and I would not like to lose her. I am sure the old man will also be very pleased to see them again. I am sorry to say Fred's children do not pay the attention to Grandpa they ought to do, and so close to one another. Last voyage, I picked up 26 passengers of a shipwrecked steamer on the coast of Newfoundland, and brought them on to Liverpool. There was a very nice letter from them in the Liverpool papers of which I may be able to send you a cutting. I do not know if you will have heard of Capt. Herron, Capt. Weaver's father in law. He died just the day before I got in and I was at his funeral. His wife died just a month before. She had sailed always with him, and all the children were born at sea.

I was glad to hear your boys are able to do a little for you, dear Minnie, for you are and always were a brave woman, I was going to say girl but those days are gone, and I'm getting quite gray and bald myself. I see you have struggled nobly, so far, and I hope you will be able to pull through. I will not forget you now and again with a little help.

Does John live far from you? I suppose his son is also quite a big man and at business. Its a long time since I have had a line from him; Kindly remember me to him.

I am now on my way to Montreal again. We generally take about 28 days on the round trip, so that I'm every month at home, although only for a short time. During winter the St Lawrence is all frozen up, and then its to Baltimore. Last voyage out I had a dreadful time among the ice fields and thought at one time I was going to lose my ship as it was so dangerous among it. At the first of the season there is always a lot about. Our people are building a lot of new boats, and I'm in hopes of getting soon back in my old trade to Baltimore for this is far too risky a trade to be in with Ice, fogs and a bad coast to make. And there are times in fact nearly every voyage while close to the coast, I have not the clothes off me for five or six days. Nanie (?) Hugh's daughter which was over on a visit sailed for Jamaica a few days before I got home. She had been for six weeks in Downpatrick. But with two babies, it can't have been much pleasure. Charlie Hugh and Henry are the only two not married now. I don't have any word of Wallace. So I suppose he is going to be an old bachelor.

Now dear Minnie, I will say good bye and will post this when I get out. Give my love to each of the boys and my niece. Tell her I wish she was nearer us to visit her cousins who grieve for her loss. God bless and comfort you. Mary desired me to give you her love and made me promise to write you going out.

With much love to yourself \\
I am your affect Brother \\
West 
\end{quotation}

The following details give an account of a typical short voyage that he made thirteen years later, while Master of the Minterne: the ship left Antwerp on 5 December 1910, and went to the ports of Huelva, Algiers, Genoa and Soulia. (At that point he was 61 and signed himself as S.M. West Croskery, but the following year he signed the ship's log as West Croskery.) On this same voyage Clara Croskery  was listed as stewardess and paid One Shilling - her address was the same as Samuel's: he had remarried after Mary's death in 1899.

The fate of the Minterne is recorded in the following: 
The Minterne: Type: Steamer; GRT 3.018 tons; built GB by Richardson, Duck and Co, Stockton. Sunk by U-Boat U-30 (Erich von Rosenberg-Grusczyski) on 3 May 1915, 50 miles off Wolf Rock, en route Cardiff-Buenos Aires, carrying coal. 2 casualties (death of two firemen) \cite{www.uboat.net/ww1/ships} As shown in the above, the Minterne was struck and sunk by a German U-boat submarine in 1915.  The crew were rescued and taken to Penzance and the newspapers wrote that Captain Croskery was the Master at the time. However, Lloyds show his appointment as Master as being terminated in 1913, and there is no record on the ship's log of him at the time of the sinking.

Lloyds Registers show him as being Master of the following ships:\cite{LloydsRegCroskery}
\begin{itemize}[nosep]
\item 1865-69 Napier (iron barque) London-New Zealand, London-San Francisco)
\item 1870-71 Whittington
\item 1871 Lady Russel
\item 1873 Bristolian (\#44103) South Americas
\item 1874 Red Gauntlet (\#48809) East Indies
\item 1875 Stentor (\#70946) China, Japan, Oriental Arch.
\item 1876-78 Dawn (\#69262) Mediterranean
\item 1878-79 Olga (\#60222) Sunk outside Sulina 1 April 1879, raised 27 May 1879.
\item 1879-82 Bessarabin (\#78733) France, Portugal, Spain, Azores, Meditarranean,
\item United States, East Indies. Collison 21 February 1880.
\item 1883 Wallachia (\#87830) Mediterranean
\item 1884-85 Bessarabin "
\item 1885-93 Wallachia Mediterranean, United States, West Indies, Gulf of Mexico, Baltic States
\item 1893 Baltimore (\#91142) United States
\item 1894-97 Rossmore (\#96336) United States, British North America, Greenland, Iceland. Collision 30 August 1895.
\item 1898-99 Tropea (\#99433) United States
\item 1901 Birdoswald (was Tropea) "
\item 1901-03 Bedouin (\#105332) East Indies
\item 1905 Inkula (\#109335) China, Japan, Oriental Arch.
\item 1908-13 Minterne (\#118349) Australia, United States, India, Burma, Mauritius
\item 1913 Upcerne (\#120694) South America. Damaged by collision 29 October 1913, "colliding vessel alone to blame".
\end{itemize}

His appointment as a Ship's Master ceased on  12 November 1913. (He was aged 63 at the time)

After he retired he lived at 9 Easton Road in New Ferry, Birkenhead, Cheshire.

He died on 26 May 1933.\cite{WestCroskeryProbate}

The Probate notice read as follows:\cite{WestCroskeryProbate}
\begin{quotation}
CROSKERY Samuel Maxwell West of 9 Easton-road New Ferry Cheshire died 26 May 1933 Probate Liverpool 11 July to Richard James Hancox bank inspector and Willian Davies Hughes estate agent.  Effects \pounds 8616 0s. 10d. Resword \pounds 8447 4s. 10d.
\end{quotation}
