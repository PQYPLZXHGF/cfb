\biohead{William Munday}{ }

William Munday was born on 7 August 1800 \cite{WillMundayBirth} in Bishopstrow, near Warminster, Wiltshire, to James Munday (\p{James_Munday}) and Jemima Browne (\p{Jemima_Browne}).  He had eight siblings:  Jemima Munday (1798-1870), Catherine Munday (1802--1883), Sarah Munday (1803--1869), James Munday (1805--1863), Mary Elizabeth Munday (1807--1896), John Munday (1809--1835), Henry Thomas Munday (1813--1895) and George Munday (1815-1830).

He married Mary Hill (\p{Mary_Hill}) on 1 December 1835 in Paulton, Somerset and they had ten children:  George Hill Munday (1836--1862), Captain James William Munday (1838-1875), Mary Elizabeth Munday (1840--1849), Anna Maria Munday (1841--1895), Sarah Adeline Munday (1843-1924), John Hill Munday (\p{John_Hill_Munday}), Thomas Hill Munday (1846--1862), Walter Edward Munday (1847-1932), Nelson Munday (1848--1886) and Louisa Fry Munday (1851--1881).

From 1837 until 1858 he was a wine merchant in Weymouth Street, Warminster, Wiltshire \cite{WillMundayOccupation}. William Cobbett wrote in 1826 in `Rural Rides' that: ``Warminster is a very nice town; everything belonging to it is solid and good.'' Despite this, they later moved to Battersea, and lived at 32 Middleton Road, where he was a wine and spirit merchant until retiring in his late sixties.\cite{WillMundayBattersea}

He died on 26 December 1886 (according to John Hill Munday's diary, "a little before 3 in the morning") and was buried at Norbiton Cemetery, Surrey on 29 December \cite{WillMundayDeath}.



