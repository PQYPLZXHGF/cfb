\biohead{Nora Katie Munday}{}

Nora was born at 7.45pm on 5 June 1881 at  Shalston Villas, Surbiton Hill,  Surbiton, Surrey to John Hill Munday (\p{John_Hill_Munday}) and Catherine Aldridge (\p{Catherine_Aldridge}), and christened on 27 July 1881 at Christchurch, Surbiton.\cite{NoraMundayBirth} 	She had four siblings: Kathleen Munday (\p{Kathleen_Munday}), Mildred Mary Munday (\p{Mildred_Mary_Munday}), Ralph Munday (\p{Ralph_Munday}) and Margery Munday(\p{Margery_Munday}).

In the years  1891 to 1901 (and beyond) the family lived at The Mendips, Langley Ave., Surbiton \cite{NoraMundayResidence}  Educated at Cheltenham Ladies College, she became an accomplished photographer before her marriage.
\cite{NoraMundayOccupation}

She married Frederick Westbrook (\p{Frederick_Westbrook}) on 8 November 1916 at the Holy Trinity Church, Sloane Street, London \cite{NoraMundayMarriage}.  He was an officer in the colonial police, and Nora went out to Ghana (then the Gold Coast) with him, one of the very few white women to do so. She appears on the passenger lists for the Abinsi, (17 July 1917), the Ekari (5 October 1921) and the Appam (4 February 1923) - with the port of departure being Lagos \cite{NoraMundayTravel}. She travelled up country with her husband and visited areas in which the native population had never seen a white woman before. As was the case with many colonial administrators, Westbrook's health was damaged by his service and he died in about 1925 after retiring and settling in Devonshire. They had no children; Nora did not remarry but lived to be 91 (as recorded in notes from her niece, Virginia Grebenik). She lived an independent life in Kensington until her death in September 1972 in 	Hammersmith, Greater London \cite{NoraMundayDeath}.

