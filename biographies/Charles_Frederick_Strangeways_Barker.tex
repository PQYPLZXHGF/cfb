\biohead{Charles Frederick Strangeways Barker}{CharlesFrederickStrangewaysBarker}{}


He was born on 21 August 1878 in	Liverpool, Lancashire to Thomas Henry Barker (1841--1917) and Mary Ellen Moulsdale (1845--1936) and christened on 30 September 1877 at St Brides, Liverpool [CFSBarkerBaptism]. He had six siblings:  James Denton Barker (1876--1958), Reverend Thomas Percy Conyers Barker (1879--1948), Francis Darcy Mead Barker (1880--1937), William Danby Holt Barker (1882--1940), Jonathan Tong Barker (1883--1950) and Henry Bertram Mitford Barker (1885-?). 


In 1901 he was an Assistant Clerk at the Liverpool  Chamber of Commerce [CFSBarkerOccupation].  By 1910, he had enlisted in the  4th Battalion, Cheshire Regiment, Reg. No. 1021 [CFSBarkerMilitary]

He married Phyllis May Wickham and they had one daughter, Peggy.


On 18 February 1930 he was (possibly) filing for bankruptcy as an Asbestos merchant in Liverpool:

" Barker Charles Frederick Strangways of Charlton, Quarry Drive, Aughton, Ormskirk, in the county of Lancaster, ASBESTOS MERCHANT and lately carrying on business at 51 Old Hall-street in the city of Liverpool.
Court - Liverpool.
No of matter - 80 of 1921
Last day for receiving proofs March 4 1930
Name of trustee and address - Allcorn James, Government Buildings, Victoria St. Liverpool Official receiver"[CFSBarkerBankruptcy Notice]


He died on 21 January 1962 at the Newsham General Hospital, Liverpool [CFSBarkerProbate].
