\biohead{Charles Edward Hancox}{}

Charles Edward Hancox was born on 15 February 1872 at 64 Woodville Terrace, Toxteth Park, Everton, Liverpool, Lancashire \cite{CEHancoxBirth},  and christened on 10 March 1872 at Holy Trinity Church, Walton Breck, Lancashire \cite{CEHancoxBaptism}. His parents were Harry Hancox (\p{Harry_Hancox}) and Maria Mary Merrett (\p{Maria_Mary_Merrett}) and he had three siblings: Harry Merret Hancox (\p{Harry_Merrett_Hancox}), Frank Heeley Hancox (\p{Frank_Heeley_Hancox}) and Richard James Hancox (\p{Richard_James_Hancox}).


On 3 April 1881 the family were living at 30 Edge Lane, Liverpool \cite{CEHancoxResidence} and by 1891 he was working as a Merchants Clerk in Toxteth Park, Liverpool, Lancashire \cite{CEHancoxOccupation1}.
By 1901 he was a manager for a Cotton Merchant, still living at home with his mother Maria and younger brothers. 

He married Alice Margaret Renner on 7 June 1905 at St Mary's, Liscard, Cheshire \cite{CEHancoxMarriage} and they had five children:  Charles Stanley Hancox ((\p{Charles_Stanley_Hancox}), Winifred Margaret Hancox (\p{Winifred_Margaret_Hancox}), Norman Merrett Hancox (\p{Norman_Merrett_Hancox}), Barbara M. Hancox (\p{Barbara_May_Hancox}) and Philip Renner Hancox (\p{Philip_Renner_Hancox}).
They lived at 54 Manor Road, Liscard, Cheshire, after their marriage\cite{CharlesEdwardHancoxHouse} and later at 13 Emmerdale Road, New Brighton, Cheshire. 

He subsequently became a cotton broker with extensive travel to the United States.  On 24 July 1916 he is recorded on the passenger list of the ship "St Paul" arriving in New York with next of kin given as Mrs. A.M. Hancox \cite{CEHancoxTravel}. On 15 March 1937 he is listed on the passenger list arriving in Southampton from Madeira, Portugal on the Vandyck (Lamport and Holt Line), and is still a cotton merchant. From the following it is evident that he had business interests in the USA where he was in partnership in a large importing cotton business in the south \cite{CEHancoxOccupation2} at least until 1920 (as seen in the following notice:
From the London Gazette, 31 August 1920:

\begin{quotation}
NOTICE is hereby given, that the Partnership heretofore subsisting between the under- signed, Frank Charles Minoprio, Charles Edward Hancox and Edward Scannell Butler, carrying on business as Cotton Merchants, at Liverpool and New Orleans, La., U..S.A., under the style or firm of MINOPRIO \& CO., and in Texas, U.S.A., under the style or firm of KENWORTiHY, MINOPRIO \& CO., has this day expired, as far as regards the said Edward Scannell Butler, who retires from the firm. All debts due to and owing by the late firm will be received and paid by the undersigned.—Dated this 31st day of August, 1920. FRANK C. MINOPRIO, C. E. HANCOX. From the London Gazette, September 1921: NOTICE is hereby given, that the Partnership heretofore subsisting between us, the under- ed, Frank Charles Minoprio, Charles Edward S1 L - - Hancox and John Louis Jones, carrying on business as Cotton Merchants and Shippers, at 39/41, Old Hall-street, Liverpool, and Houston, Texas, U.S.A., under the style or firm of MINOPRIO \& CO., has been dissolved as and from the 31st day of August, 1921. All debts due to and owing by the said firm will be received and paid in Liverpool by the said F. O. Minoprio and C. E. Hanoox, and in America by the said J. L. Jones.—Dated this 31st day of August, 1921. FRANK C. MINOPRIO. C. E, HANCOX (for J. L. Jones). 086 C. E. HANCOX. September, 1921.
\end{quotation}

 By 1938 his address was given as Witley Court, 54--64 Coram Street, Holborn, London.\cite{CharlesEdwardHancoxLondonhouse}  The family also spent all their summer holidays in North Wales, where they had a house at Abersoch in north Wales.

He died in the first quarter of 1952 in the Wirral, Cheshire \cite{CEHancoxDeath}.
